\section{Motivation}
\label{sec:motivation}

Einer der Hauptzwecke des Internets of Things (IoT) besteht darin, physische Gegenstände über das Internet zu verbinden. Ein Hauptmerkmal des Industrial Internet of Things (IIoT) ist dann im Weiterführenden, verschiedene industrielle Geräte auf breiter Ebene miteinander zu verknüpfen. Zusammen mit intelligenter Software ist damit die Entwicklung immer fortschrittlicherer technologischer Fertigungssysteme, die über Softwareprogramme aus der Ferne überwacht und gesteuert werden können, möglich. Die Adaption solcher Technologien ist noch recht niedrig, was dadurch verdeutlicht wird, dass zum Stand von 2020 weniger als 30 \% der Hersteller von einer umfassenden Adaption berichten [\cite{industrie40adaption}]. Ein Grund für diese langsame Umstellung liegt an der hohen Komplexität der eingesetzten Softwarekomponenten. Oft werden spezialisierte Einzellösungen genutzt, welche in der Regel bereits während des Planungs- und Aufbauprozesses der Anlage bedacht und integriert werden müssen. [\cite{industrie40slowAdoption}]

Diese Arbeit beschäftigt sich mit interoperablen, modularen und benutzerfreundlichen Softwarelösungen, derren Ziel es ist, eine an Industrie 4.0 und IIoT angepasste Prozessplanung und -steuerung sowie Produktionsplanung zu ermöglichen. Dabei werden Applikationen in den genannten Bereichen konzeptuell implementiert und ihr Potenzial für die Industrie wird analysiert.
