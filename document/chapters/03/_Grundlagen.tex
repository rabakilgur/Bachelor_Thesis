\chapter{Grundlagen}
\label{cha:grundlagen}

Die hier aufgeführten und beschriebenen Technologien bilden die Grundlage der in dieser Arbeit entworfenen Anwendungen. Sie werden daher in den folgenden (Unter-)Kapiteln referenziert und hier nur grob in ihren Anwendungskontext eingeordnet.\\
Nach einem Überblick über HTTP und HTTPS (Kapitel \ref{sec:http}) und damit der fundamentalen und traditionellen Kommunikation im Internet folgt eine Zusammenfassung zu REST und REST APIs (Kapitel \ref{sec:rest}), der vorherrschenden Technologie für Programmschnittstellen im Internet, und JSON (Kapitel \ref{sec:json}), dem prävalent Datenformat für REST APIs. Darauf folgt ein kurzer Einblick in OPC-UA (Kapitel \ref{sec:opcua}) mit einem besonderen Fokus auf die Unterschiede zu REST, und ein Einblick in das Web Socket Protokoll (Kapitel \ref{sec:websockets}), welches eine mächtige Erweiterung zu üblichen REST APIs darstellen kann. Abschließend wird noch auf UUIDs (Kapitel \ref{sec:uuid}) eingegangen, welche in vielen Fällen als Identifikatoren benutzt werden.

Es sei auch erwähnt, dass in den folgenden Kapiteln die Begriffe „Internet“ und „World Wide Web“ (WWW, oder kurz „Web“) synonym verwendet werden.

\subimport*{}{1_HTTPS}
\subimport*{}{2_REST}
\subimport*{}{3_JSON}
\subimport*{}{4_OPC-UA}
\subimport*{}{5_WebSockets}
\subimport*{}{6_UUID}
