\chapter{Endbenutzergesteuerte Prozessplanung und -steuerung \& automatisierte und anpassbare Produktionsplanung}
\label{cha:prozessplanungUndProzesssteuerung}

Im Hinblick auf die Zielsetzung dieser Arbeit wird sich in diesem Kapitel der Entwicklung einer prototypischen Softwarelösung zur (end-)benutzergesteuerten Prozessplanung sowie -steuerung angenommen. Diese Aufgabe lässt sich in zwei offensichtliche Teilaufgaben aufteilen:

\begin{enumerate}
    \item \textbf{Prozessplanung} – Ein benutzerfreundliches, simples und dennoch mächtiges Werkzeug zur feingranularen Planung von Fertigungsprozessen
    \item \textbf{Prozesssteuerung} – Ein an die Prozessplanung angebundenes System zur Übermittlung und Steuerung der Maschinen einer verbundenen Fertigungsanlage
\end{enumerate}

Die konkrete Software, welche hier entwickelt, betrachtet und analysiert wird, ist Teil der Digital Twin Platform des DiK. Genauer soll sie über die Digital Twin Academy Webseite unter dem Namen „Flexible MBOM Generation“ bereitgestellt werden. Bei dieser Webseite handelt es sich um eine didaktische Plattform, welche den Besuchern die Implementierung von IIoT näher bringt und ihnen ermöglicht, direkt und auf verschiedenen Ebenen mit Maschinen in verteilten Standorten zu interagieren.

Bevor im Folgenden dann auf die Konzeption und Entwicklung sowie das weitere Potenzial dieser Softwarelösung(en) eingegangen wird, wird zunächst ein Blick auf die „OPC Factory“ Platform geworfen. Kapitel \ref{sec:prozessplanung} beschäftigt sich dann zunächst mit der Benutzeroberfläche und dem Interaktionsaspekt der Applikation, während das darauffolgende Kapitel \ref{sec:prozesssteuerung} sich näher mit der Kommunikations-Infrastruktur beschäftigt. Zuletzt wird in Kapitel \ref{sec:produktionsplanung} ein Konzept für eine automatisierte und anpassbare Produktionsplanungs-Applikation entworfen.\\
Die Implementierung dieser Anwendungen findet dann in Kapitel \ref{sec:implementierung} statt.

\subimport*{}{1_OPC-Factory}
\subimport*{}{2_Prozessplanung}
\subimport*{}{3_Prozesssteuerung}
\subimport*{}{4_Produktionsplanung}
