\section{Potenzialanalyse}
\label{sec:potenzialanalyse}

Wie bereits in Kapitel \ref{sec:motivation} angesprochen wurde, ist die Verbreitung von Industrie 4.0 und IIoT Methoden in der Industrie verbesserungsfähig. Die vorgestellten Softwarelösungen haben zum einen dank ihrer Eigenschaften wie Interoperabilität, Serviceorientierung und Modularität die Möglichkeit, auf breiter Fläche eingesetzt werden zu können und so die Vorteile und Potenziale von Industrie 4.0 und IIoT zu weiteren Anbietern zu bringen; zum anderen haben diese Applikationen als Teil der Digital Twin Academy ein didaktisches Potenzial, da die Plattform diese Themenbereiche Benutzern aus der Industrie und dem akademischen Bildungsbereich näher bringt und so versucht, eine größere Adaption und Wahrnehmung dieser Technologien zu erreichen.

\subsection*{Potenzialanalyse der Prozessplanung und -steuerung}

Nun soll ein näherer Blick auf die endbenutzergesteuerte Prozessplanung und -steuerung geworfen werden. Diese Anwendung erweitert die für Industrie 4.0 und IIoT typischen Potenzialen, da sie bestehende Anlagen und Fertigungsstätten auf moderne Gegebenheiten wie Heimarbeit oder Fernwartung vorbereitet.

Ein weiterer Anwendungsfall einer solchen Applikation ist aufgrund ihrer niedrigen Eintrittsbarriere das Manufacturing-as-a-Service (MaaS). MaaS bezeichnet die gemeinsame Nutzung einer vernetzten Fertigungsinfrastruktur zur Herstellung von Produkten, wobei diese Nutzung als Service angeboten wird. 

Zuletzt sei noch die Gesamtarchitektur in den Fokus gehoben. REST APIs und OPC-UA sind beide vorherrschende Kommunikationsprotokolle, welche sich in verschiedenen Bereichen etabliert haben, wobei in der Fertigungs-Industrie in den meisten Fällen OPC-UA für die Kommunikation genutzt wird, insbesondere zu Maschinen. IoT-Geräte nutzen jedoch üblicherweise REST APIs für ihre Kommunikation. Eine Verbindung beider Protokolle ist hier nicht nur sinnvoll, sondern es hilft auch dabei, diese beiden genannten Bereiche einander näherzubringen – eine Aufgabe, welche zentraler Bestandteil von Industrie 4.0  und IIoT ist.

\subsection*{Potenzialanalyse der Produktionsplanung}

Das Potenzial betreffend ist diese Applikation wie zu erwarten analog zu der Softwarelösung zur Prozessplanung und -steuerung (siehe Kapitel \ref{sec:potenzialanalyse}). Aufgrund des ähnlichen Anwendungsfalles gelten die dort genannten Punkte also auch hier.
