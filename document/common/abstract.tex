% =================================================================================================== %
% Abstract
% =================================================================================================== %
\hfill
\section*{Kurzfassung}
Seit nunmehr einigen Jahren ist ein zunehmendes Aufkommen von allgegenwärtigen Verknüpfungen mit digitalen Objekten unter dem Namen „Internet of Things“ (IoT; auch bekannt als Internet of Objects) zu bemerken. Historisch waren Industrial Automation and Control Systems (IACS, dt.: Industrielle Automations- und Kontrollsysteme) jedoch größtenteils von einer solchen Verknüpfung ausgenommen. Um diese Lücke zu schließen, ist in jüngerer Vergangenheit das Konzept des Industrial Internet of Things (IIoT) aufgekommen. Dessen Ziel ist es, eine industrielle Infrastruktur über ähnliche Methoden wie beim IoT anzubinden.

Diese Thesis beschäftigt sich mit dem daraus hervorgehenden Potenzial der endbenutzergesteuerten Prozesssteuerung, -analyse, und -planung, sowie mit den Herausforderungen, die dadurch entstehen. Dabei wird untersucht, inwiefern diese Aspekte vereinfacht und individualisiert werden können. Des Weiteren werden webbasierte und interoperable Software-Komponenten entwickelt, in eine Demonstrationsplattform eingebunden und analysiert.

\textbf{Schlüsselwörter:} Industrie 4.0, Industrial Internet of Things (IIoT), Prozesssteuerung, Produktionsplanung, REST, OPC-UA, Manufacturing Bill of Materials (MBOM)

%\vspace{15mm}
\newpage
\hfill

\selectlanguage{english}
\section*{Abstract}
For some years now, one has noticed an increasing emergence of ubiquitous connections with digital objects under the name „Internet of Things“ (IoT; also known as Internet of Objects). Historically, Industrial Automation and Control Systems (IACS) were largely excluded from such a link. To close this gap, the concept of the Industrial Internet of Things (IIoT) has recently emerged. It aims to connect an industrial infrastructure using methods similar to those used for the IoT.

This thesis deals with the resulting potential of end-user-controlled process control, analysis and planning, as well as with the challenges that arise from this. It is examined to what extent these aspects can be simplified and individualized. Furthermore, web-based and interoperable software components are developed, integrated into a demonstration platform, and analyzed.

\textbf{Keywords:} Industrie 4.0, Industrial Internet of Things (IIoT), Process Control, Production Planning, REST, OPC-UA, Manufacturing Bill of Materials (MBOM)
\selectlanguage{ngerman}
% =================================================================================================== %
